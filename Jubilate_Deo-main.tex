
\documentclass[12pt, a5paper]{book}
%\usepackage{luaotfload}
\usepackage{palatino}
\usepackage{tgtermes}
\usepackage[T1]{fontenc}
\usepackage{tgothic}
%\usepackage[utf8]{inputenc}
\usepackage{fullpage}
\usepackage{graphicx}
\usepackage{gregoriotex}
\usepackage{gregoriosyms}
\pagestyle{empty}

\input jubilate_deo-conf.tex

\begin{document}

\def\greinitialformat#1{%
{\fontsize{43}{43}\selectfont #1}%
}

\setgrefactor{17}

\begin{center}
%{\fontfamily{fvm}\fontseries{b}\fontsize{24}\selectfont Jubilate Deo}\\
%{\usefont{OT1}{tgoth}{m}{n} 
\texttgoth{\Huge Jubilate Deo\\}
%{\normalfont\tgothfamily\Huge Jubilate Deo}\\
\texttgoth{\Large Reproduction of the Typis Polyglottis\\ Vaticanis Edition}\\
\vspace{1in}

{\large Issued by the Sacred Congregation for Divine Worship as a "personal gift of Paul VI to Catholic Bishops of the world and the heads of religious orders, April 14, 1974}\\

%\includegraphics{01.eps}
\end{center}
\newpage
\begin{center}
\texttgoth{\Large Voluntati Obsequens}\\
\vspace{5mm}
Letter to the Bishops on the Minimum Repertoire of Plainchant
\end{center}

This mimimum repertoire of Gregorian chant has been prepared with that purpose in mind: to make it easier for Christians to achieve unity and spiritual harmony with their brothers and with the living traditions of the past. Hence it is that those who are trying to improve the quality of congregational singing cannot refuse to Gregorian chant the place which is due to it. And this becomes all the more imperative as we approach the Holy Year of 1975, during which the faithful of different languages, nations and origins, will find themselves side by side for the common celebration of the Lord....
\par
In presenting the Holy Father's gift to you, may I at the same time remind you of the desire which he has often expressed that the Conciliar constitution on the liturgy be increasingly better implemented. Would you therefore, in collaboration with the competent diocesan and national agencies for the liturgy, sacred music and catechetics, decide on the best ways of teaching the faithful the Latin chants of "{\it Jubilate Deo}" and of having them sing them, and also of promoting the preservation and execution of Gregorian chant in the communities mentioned above. You will thus be performing a new service for the Church in the domain of liturgical renewal. The contents of this booklet may be reproduced free of charge...
\newpage

\begin{center}
I\\CANTUS MISSAE
\end{center}

\newpage

\setlength\parindent{0cm}

\titre{AD RITUS INITIALES}

\setfirstlineaboveinitial{\small \textsc{\textbf{3}}}{\small \textsc{\textbf{3}}}
\commentary{{\small \emph{ed. Vat. XVI}}}
\includescore{kyrie.tex}

\setfirstlineaboveinitial{\small \textsc{\textbf{5}}}{\small \textsc{\textbf{5}}}
\commentary{{\small \emph{ed. Vat. VIII}}}
\includescore{gloria.tex}

\newpage
\titre{AD LITURGIAM VERBI}

\noinitial
Post lectionem I:\\
\includescore{verbum1.tex}

\vspace{2mm}
Post lectionem II vel unicam ante evangelium:\\
\includescore{verbum2.tex}

\vspace{2mm}
Alleluia:\\
\includescore{alleluia1.tex}

\vspace{2mm}
Vel:\\
\includescore{alleluia2.tex}

\vspace{2mm}
Ante evangelium:\\
\includescore{dominus_vobiscum.tex}

\vspace{2mm}
Post evangelium:\\
\includescore{verbum_domini__laus.tex}

\reinitial
\titre{CREDO}
\setfirstlineaboveinitial{\small \textsc{\textbf{5 }}}{\small \textsc{\textbf{5 }}}
\commentary{{\small \emph{ed. Vat. III}}}
\includescore{credo.tex}

\vspace{4mm}
Ad orationem universalem, post unamquamque intentionem:\\
\includescore{intentionem.tex}

\vspace{4mm}
\noinitial
\titre{AD PRECEM EUCHARISTICAM}

Ante pr\ae fationem:\\
\includescore{dominus_vobiscum3.tex}

\vspace{5mm}
\commentary{{\small \emph{ed. Vat. XVIII}}}
\includescore{sanctus.tex}

\vspace{4mm}
Acclamatio post consecrationem:\\
\includescore{mysterium.tex}

\vspace{2mm}
Post doxologiam:\\
\includescore{per_omnia.tex}

\titre{AD RITUS COMMUNIONIS}
Oratio dominica\\
\includescore{praeceptis.tex}

\includescore{pater_noster.tex}

\vspace{2mm}
Acclamatio post Libera nos:\\
\includescore{pater_noster2.tex}

\vspace{2mm}
Ad pacem:\\
\includescore{qui_vivis.tex}
\includescore{pax_domini.tex}

\scoreheader{}{ed. Vat. XVIII}
\includescore{agnus_dei.tex}

\titre{AD RITUS CONCLUSIONIS}
\vspace{2mm}
Ad dimittendum populum post benedictionem:\\
\includescore{ite1.tex}
\Rbar . De- o grá- ti- as.\\

\vspace{2mm}
Dominica Tesurrectionis, infra octavam Paschae necnon dominica Pentecostes:\\
\includescore{ite2.tex}
\Rbar . De- o grá- ti- as, alle- lú- ia, alle- lú- ia.\\

\newpage
\titre{II\\CANTUS VARII}
\newpage

\titre{O SALUTARIS}
\chant{8}{}{o_salutaris.tex}
\pagebreak
\titre{ADORO TE}
\chant{}{}{adoro_te.tex}
Mulla fit pausa in quarto versu stropharum 2 et 6.\\\\
\begin{tabular}{ll}
2 &Visus, tactus, gustus in te fállitur;\\
&sed audítu solo tuto créditur.\\
&Credo quidquid dixit Dei Fílius:\\
&nil hoc verbo veritátis vérius.\\
\end{tabular}
\\
\begin{tabular}{ll}
3 &In cruce latébat sola Déitas;\\
&at hic latet simul et humánitas.\\
&Ambo tamen credens atque cónfitens\\
&peto quod petívit latro paénitens.\\
\end{tabular}
\\
\begin{tabular}{ll}
4 &Plagas, sicut Thomas, non intúeor;\\
&Deum tamen meum te confíteor.\\
&Fac me tibi semper magis crédere,\\
&in te spem habére, te dilígere.\\
\end{tabular}
\\
\begin{tabular}{ll}
5 &O memoriále mortis Dómini,\\
&Panis vivus vitam pr\ae stans hómini,\\
&praésta me\ae \ menti de te vívere,\\
&et te illi semper dulce sápere.\\
\end{tabular}
\\
\begin{tabular}{ll}
6 &Pie pelicáne, Iesu Dómine,\\
&me immúndum munda tuo sánguine,\\
&cuius una stilla salvum fácere\\
&totum mundum quit ab omni scélere.\\
\end{tabular}
\\
\begin{tabular}{ll}
7 &Iesu, quem velátum nunc aspício,\\
&oro fiat illud quod tam sítio;\\
&ut, te reveláta cernens fácie,\\
&visu sim beátus tu\ae \ glóri\ae . Amen.
\end{tabular}

\titre{TANTUM ERGO}
\chant{3}{}{tantum_ergo.tex}


\titre{Psalmus 116}
\titre{LAUDATE}
\chant{6}{}{laudate.tex}
\par
\begin{tabular}{ll}
2&Quoniam confirm\'ata est super nos miseric\'ordi{\it a} {\bf e}ius, * et véritas Dómini manet {\it in ae}{\bf tér}num.\\
3&Glória Patri, {\it et} {\bf Fí}lio, * et Spirí{\it tui} {\bf Sanc}to.\\
4&Sicut erat in princípio, et nunc {\it et} {\bf sem}per, * et in saécula saecu{\it lorum.} {\bf A}men.\\
\end{tabular}

\titre{PARCE DOMINE}
\chant{1}{}{parce_domine.tex}

\titre{DA PACEM}
\chant{2}{}{da_pacem.tex}

\titre{UBI CARITAS}
\chant{}{}{ubi_caritas.tex}

\Vbar . Congregávit nos in unum Christi amor.\\
\Vbar . Exsultémus et in ipso iucundémur.\\
\Vbar . Timeámus et amémus Deum vivum.\\
\Vbar . Et ex corde diligámus nos sincéro.\\
\\
Ant. Ubi cáritas est vera, Deus ibi est.\\
\\
\Vbar . Simul ergo cum in unum congregámur:\\
\Vbar . Ne nos mente dividámur, caveámus.\\
\Vbar . Cessent iúrgia malígna, cessent lites.\\
\Vbar . Et in médio nostri sit Christus Deus.\\
\\
Ant. Ubi cáritas est vera, Deus ibi est.\\
\\
\Vbar . Simul quoque cum beátis videámus\\
\Vbar . Gloriánter vultum tuum, Christe Deus:\\
\Vbar . Gáudium, quod est imménsum atque probum,\\
\Vbar . Saécula per infiníta s\ae culórum.\\
\\
Ant. Ubi cáritas est vera, Deus ibi est.\\

\titre{VENI CREATOR}
\chant{8}{}{veni_creator.tex}

\begin{tabular}{ll}
2 &Qui díceris Paráclitus,\\
&altíssimi donum Dei,\\
&fons vivus, ignis, cáritas\\
&et spiritális únctio.\\
\\
3 &Tu septifórmis múnere,\\
&díg{\it i}tus patérn\ae déxteráe ,\\
&tu rite promíssum Patris\\
&sermóne ditans gúttura.\\
\\
4 &Accénde lumen sénsibus,\\
&infúnd{\bf e} amórem córdibus,\\
&infírma nostri córporis\\
&virtúte firmans pérpeti\\
\\
5 &Hostem repéllas lóngius\\
&pacémque dones próntius;\\
&ductóre sic te praévio\\
&vitémus omne nóxium.\\
\\
6 &Per te sciámus da Patrem\\
&noscámus atque Fílium,\\
&tequ{\it e} utriúsque Spíritum\\
&credámus omni témpore.\\
&\ \ \ \ \ Amen.
\end{tabular}

\Vbar . Emítte Spíritum tuum et creabuntur.\\
\Rbar . Et renovábis fáciem terr\ae .

Orémus.\\
\vspace{2mm}
Deus, qui corda fidélium Sancti Spíritus illustratióne docuísti, \dag da nobis in eódem Spíritu recta sápere, * et de eius semper consolatióne gaudére. Per Christum Dóminum nostrum. \Rbar. Amen.

\titre{REGINA C\AE LI}
\chant{6}{}{regina_caeli.tex}

\titre{SALVE REGINA}
\chant{5}{}{salve_regina.tex}

\titre{AVE MARIS STELLA}
\chant{1}{}{ave_maris_stella.tex}

\titre{MAGNIFICAT}
\chant{6}{}{magnificat.tex}

%columns appear to be 2.0 and 3.4. Use \un and \deux
\begin{tabular}{rl}
3 &{\it Quia} respéxit humilitátem ancíll{\it \ae}  {\bf su}\ae, * ecce enim ex hoc beátam me dicent omnes gene{\it rati}{\bf ó}nes.\\
4&{\it Quia} fecit mihi magna, {\it qui} {\bf po}tens est, * et sanctum {\it nomen } {\it e}ius.\\
5&{\it Et mi}sericórdia eius a progénie in {\it pro}{\bf gé}nies * timen{\it tibus }{\it e}{\bf um}.\\
6&{\it Fecit} poténtiam in brácchi{\it o }{\bf su}o, * dispérsit supérbos mente {\it cordis} {\it su}i.\\
7&{\it Depó}suit poténtes {\it de} {\bf se}de, * et exal{\it távit} {\bf hú}miles.\\
8&{\it Esu}riéntes implé{\it vit} {\bf bo}nis, * et dívites dimí{\it sit in}{\bf á}nes.\\
9&{\it Suscé}pit Israel, púe{\it rum} {\bf su}um, * recordátus misericór{\it di\ae } {\it su}\ae.\\
10&{\it Sicut} locútus est ad pa{\it tres} {\bf no}stros, * Abraham et sémini e{\it ius in }{\bf saé}cula.\\
11&{\it Glória} Patri, {\it et} {\bf Fí}lio, * et Spirí{\it tui} {\bf Sanc}to.\\
12&{\it Sicut} erat in princípio, et nunc {\it et} {\bf sem}per, * er in saécula s\ae cul{\it órum}. {\bf A}men.
\end{tabular}

\titre{TU ES PETRUS}
\chant{7}{}{tu_es_petrus.tex}

\titre{HYMNUS TE DEUM}
\chant{3}{}{te_deum.tex}

\Vbar . Benedicámus Patrem et Fílium cum Sancto Spíritu.\\
\Rbar . Laudémnus et superexaltémus eum in saécula.\\
\vspace{2mm}
\Vbar . Bedíctus es, Dómine, in firmaménto c\ae li.\\
\Rbar . Et laudábilis, et gloriósus, et superexaltátus in saécula.\\
\vspace{2mm}
\Vbar . Dómine, exáudi oratiónem meam.\\
\Rbar . Et clamor meus ad te véniat.\\
\vspace{5mm}

Orémus.
\vspace{2mm}
Deus, cuius misericórdi\ae non est númerus et bonitátis infinítus est thesáurus, \dag piíssim\ae  maiestáti tu\ae  pro collátis donis grátias ágimus, tuam semper cleméntiam exorántes, * ut qui peténtibus postuláta concédis, eósdem non déserens, ad praémia futúra dispónas. Per Christum Dóminum nostrum. \Rbar . Amen.


\end{document}



